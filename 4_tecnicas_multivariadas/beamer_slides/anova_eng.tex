\documentclass{beamer}
\usepackage[utf8]{inputenc}

\title{Analysis of Variance (ANOVA)}
\author{Your Name}
\date{\today}

\begin{document}

\begin{frame}
\titlepage
\end{frame}

\begin{frame}{Table of Contents}
    \tableofcontents
\end{frame}

\section{Introduction}
\begin{frame}{Introduction}
    \begin{itemize}
        \item Analysis of variance (ANOVA) is a statistical technique used to determine if there are significant differences between the means of two or more groups.
        \item ANOVA decomposes the total variability in the data into variability between groups and variability within groups.
        \item The F statistic is used to test if the group means are equal.
    \end{itemize}
\end{frame}

\section{Equations}
\begin{frame}{Equations}
    \begin{itemize}
        \item The total sum of squares (SST) is calculated as: $SST = \sum_{i=1}^{n}(y_i-\overline{y})^2$
        \item The between-group sum of squares (SSB) is calculated as: $SSB = \sum_{j=1}^{k}n_j(\overline{y_j}-\overline{y})^2$
        \item The within-group sum of squares (SSW) is calculated as: $SSW = \sum_{j=1}^{k}\sum_{i=1}^{n_j}(y_{ij}-\overline{y_j})^2$
    \end{itemize}
\end{frame}

\section{Sum of Squares Decomposition}


\begin{frame}{Sum of Squares Decomposition}
  \begin{itemize}
      \item The total sum of squares (SST) can be decomposed into the between-group sum of squares (SSB) and the within-group sum of squares (SSW).
      \item The equation for this decomposition is: $SST = SSB + SSW$
      \item The F statistic is calculated as the ratio of the mean square between groups (MSB) to the mean square within groups (MSW).
      \item The equation for the F statistic is: $F = \frac{MSB}{MSW}$, where $MSB = \frac{SSB}{k-1}$ and $MSW = \frac{SSW}{N-k}$
  \end{itemize}
\end{frame}

\section{Relation with Linear Model}
\begin{frame}{Relation with Linear Model}
  \begin{itemize}
      \item ANOVA can be seen as a special case of the linear model.
      \item The linear model for ANOVA is: $y_{ij} = \mu + \alpha_j + \epsilon_{ij}$, where $\mu$ is the overall mean, $\alpha_j$ is the effect of group $j$, and $\epsilon_{ij}$ is the error term.
      \item The null hypothesis for ANOVA is that all group means are equal, which can be written as: $H_0: \alpha_1 = \alpha_2 = ... = \alpha_k = 0$
      \item The alternative hypothesis is that at least one group mean is different from the others.
  \end{itemize}
\end{frame}

\section{Numerical Example}
\begin{frame}{Numerical Example}
  \begin{itemize}
      \item Suppose we have three groups with the following data:
      \begin{itemize}
          \item Group 1: 6, 8, 7
          \item Group 2: 9, 10, 11
          \item Group 3: 12, 13, 14
      \end{itemize}
      \item The overall mean is $\overline{y} = 10$
      \item The group means are $\overline{y_1} = 7$, $\overline{y_2} = 10$, and $\overline{y_3} = 13$
      \item The SST is calculated as: $SST = (6-10)^2 + (8-10)^2 + ... + (14-10)^2 = 60$
      \item The SSB is calculated as: $SSB = 3(7-10)^2 + 3(10-10)^2 + 3(13-10)^2 = 54$
      \item The SSW is calculated as: $SSW = (6-7)^2 + (8-7)^2 + ... + (14-13)^2 = 6$
  \end{itemize}
\end{frame}

\begin{frame}{Numerical Example (cont.)}


  \begin{itemize}
    \item The F statistic is calculated as: $F = \frac{MSB}{MSW} = \frac{\frac{SSB}{k-1}}{\frac{SSW}{N-k}} = \frac{\frac{54}{3-1}}{\frac{6}{9-3}} = 9$
    \item The p-value is calculated by comparing the F statistic to an F distribution with $(k-1)$ and $(N-k)$ degrees of freedom.
    \item In this case, the p-value is very small, indicating that we can reject the null hypothesis and conclude that there are significant differences between the group means.
\end{itemize}
\end{frame}

\section{Interpretation of Results}
\begin{frame}{Interpretation of Results}
\begin{itemize}
    \item If the p-value is smaller than the significance level (usually 0.05), we reject the null hypothesis and conclude that there are significant differences between the group means.
    \item If the p-value is larger than the significance level, we fail to reject the null hypothesis and conclude that there is not enough evidence to suggest that there are significant differences between the group means.
    \item It is important to note that ANOVA only tells us if there are significant differences between the group means, but it does not tell us which specific means are different from each other.
    \item To determine which specific means are different from each other, we can use post-hoc tests such as Tukey's HSD test or Bonferroni's correction.
\end{itemize}
\end{frame}

\end{document}