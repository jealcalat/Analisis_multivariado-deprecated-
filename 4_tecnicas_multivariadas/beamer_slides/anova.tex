\documentclass{beamer}
\usepackage[utf8]{inputenc}
\usepackage[spanish]{babel}

\title{Análisis de varianza de una vía}
\author{Tu nombre}
\date{\today}

\begin{document}

\begin{frame}
\titlepage
\end{frame}

\begin{frame}{Contenido}
    \tableofcontents
\end{frame}

\section{Introducción}
\begin{frame}{Introducción}
    \begin{itemize}
        \item El análisis de varianza (ANOVA) es una técnica estadística que se utiliza para determinar si hay diferencias significativas entre las medias de dos o más grupos.
        \item En el caso del ANOVA de una vía, se analiza la varianza entre los grupos y dentro de los grupos para determinar si hay diferencias significativas entre las medias de los grupos.
    \end{itemize}
\end{frame}

\section{Aspectos matemáticos y ecuaciones}
\begin{frame}{Aspectos matemáticos y ecuaciones}
    \begin{itemize}
        \item La hipótesis nula del ANOVA de una vía es que todas las medias de los grupos son iguales.
        \item La hipótesis alternativa es que al menos una media es diferente.
        \item Para realizar el ANOVA de una vía, se calcula la suma de cuadrados total (SST), la suma de cuadrados entre grupos (SSB) y la suma de cuadrados dentro de los grupos (SSW).
        \item La fórmula para calcular la SST es: $SST = \sum_{i=1}^{k}\sum_{j=1}^{n_i}(x_{ij}-\overline{x})^2$
        \item La fórmula para calcular la SSB es: $SSB = \sum_{i=1}^{k}n_i(\overline{x_i}-\overline{x})^2$
        \item La fórmula para calcular la SSW es: $SSW = \sum_{i=1}^{k}\sum_{j=1}^{n_i}(x_{ij}-\overline{x_i})^2$
    \end{itemize}
\end{frame}

\section{Interpretación de resultados}

\begin{frame}{Interpretación de resultados}
  \begin{itemize}
      \item Una vez que se han calculado la SST, SSB y SSW, se puede calcular el estadístico F y el valor p para determinar si hay diferencias significativas entre las medias de los grupos.
      \item El estadístico F se calcula como: $F = \frac{MSB}{MSW}$, donde $MSB = \frac{SSB}{k-1}$ y $MSW = \frac{SSW}{N-k}$
      \item El valor p se obtiene comparando el estadístico F con una distribución F con $(k-1)$ y $(N-k)$ grados de libertad.
      \item Si el valor p es menor que el nivel de significancia (generalmente 0.05), se rechaza la hipótesis nula y se concluye que hay diferencias significativas entre las medias de los grupos.
  \end{itemize}
\end{frame}

\end{document}
